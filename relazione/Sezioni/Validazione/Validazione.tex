Il controllo dell'accessibilità deve essere effettuato attraverso tool automatici, semiautomatici ma anche attraverso una validazione manuale.\\
É necessario ricordare come gli strumenti automatici non siano esaustivi ed il loro utilizzo non implica che il sito sia accessibile. Essi dunque possono essere utilizzati come strumento di supporto.\\
Per la validazione dell'accessibilità del sito si è deciso di utilizzare lo strumento \textit{\href{https://web.math.unipd.it/accessibility-dev/}{myWCAG4All}}. Il tool permette di avere un elenco di test esaustivo da effettuare, oltre alla possibilità di ottenere un report finale con i test effettuati e superati.\\
L'account utilizzato per tenere traccia dei test è: \textit{alecava41programmer@gmail.com}. All'interno dell'account è possibile trovare il sito "\textit{Adrenaline Motocross Park}" e i relativi test effettuati.\\
I tool utilizzati sono molteplici, in quanto ci si è attenuti alla guida che veniva data nel sito \textit{myWCAG4All} in relazione al test che si doveva effettuare. Quelli utilizzati principalmente sono stati:
\begin{itemize}
\setlength\itemsep{0em}
\item \href{https://wave.webaim.org/}{WAVE}.
\item \href{https://www.totalvalidator.com/}{Total Validator}.
\item \href{https://www.webfx.com/tools/read-able/}{Readability Test}.
\item \href{https://validator.w3.org/}{HTML Validator}.
\end{itemize}



