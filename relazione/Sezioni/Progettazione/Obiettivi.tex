Nello sviluppo del sito il gruppo si è imposto alcuni intransigenti obiettivi.\\
In particolare il sito deve poter essere fruibile agevolmente dal maggior numero utenti possibile, compresi quelli con gravi disabilità visive e/o
motorie. Alcune contromisure significative adottate sono:
\begin{itemize}
	\item Lettura corretta delle tabelle da parte di dispositivi di assistenza (screen reader, ...);
	\item Testo alternativo per le immagini di contenuto;
	\item Testi e link con buoni livelli di contrasto;
\end{itemize}

\textbf{NB}: icone e immagini di presentazione, poiché decorative, non hanno alcun testo alternativo dato che, se venissero rimosse, l'utente capirebbe lo stesso ciò di cui si sta parlando.

L'accessibilità deve essere garantita indipendentemente dal tipo di dispositivo. Deve essere preferito lo sviluppo \textit{"Mobile First"}, data l'utenza relativamente giovane e il tipo di operazioni da eseguire (prenotazioni). È essenziale garantire all'utente la possibilità di svolgere tutte le operazioni da smartphone, al fine di rendere l'esperienza utente il più agevole possibile.\\[0.2cm]
