Il progetto \textit{Adrenaline Motocross Park}, svolto per il concorso Accattivante Accessibile, propone di implementare un sito web adibito a facilitare la gestione delle prenotazioni dei servizi offerti dall'impianto sportivo ai clienti.

\textit{Adrenaline Motocross Park} è un impianto sportivo per la pratica del motocross, aperto ai piloti tra i 10 e 50 anni di età. È considerato uno degli impianti più all'avanguardia d'Italia, grazie alle sue numerose piste e all'alta qualità dei servizi offerti. I piloti possono usufruire di una vasta gamma di tracciati di diverse difficoltà, in modo tale da accontentare sia il neofita che il professionista. Vi è anche la possibilità di effettuare dei corsi di guida con istruttori professionali. Un altro punto di forza è sicuramente il noleggio delle moto e attrezzatura, un ottimo modo per offrire la possibilità a chi non conosce questo sport di provarlo. Oltre ai servizi fondamentali già elencati, l'impianto mette a disposizione dei clienti una serie di servizi accessori.

Vista l'importante affluenza di piloti all'interno dell'impianto e la conseguente difficoltà per la direzione di tenere traccia delle prenotazioni, i gestori hanno deciso di creare un sito web per la prenotazione di ingressi, corsi e noleggi. Si considera questa soluzione molto efficiente ed efficacie, in quanto può potenzialmente ridurre di molto le code all'ingresso, agevolando il lavoro degli amministratori e migliorando l'esperienza complessiva del cliente.

Per conseguire lo scopo stabilito, si è deciso di creare qualcosa di elegante ed efficace allo stesso tempo, in modo da garantire all'utente un'esperienza piacevole non solo in pista, ma anche online. Per i frequentatori del sito è possibile registrarsi e creare il proprio account personale, grazie al quale essi possono prenotare:
\begin{itemize}
\item Ingressi presso l'impianto;
\item Un posto per i corsi ai quali è interessato a partecipare;
\item Noleggio della moto in una data in cui ha prenotato un ingresso o un corso.
\end{itemize}

Oltre alla funzione prettamente gestionale, il sito deve offrire tutte le informazioni che possono essere d'interesse per il cliente:
\begin{itemize}
\item Tracciati: informazioni sul tracciato e orari di apertura;
\item Date di apertura: prossime date in cui l'impianto sarà aperto;
\item Corsi: prossimi corsi organizzati nell'impianto;
\item Servizi: gamma di servizi aggiuntivi offerti dall'impianto;
\item Contatti: recapiti dell'impianto.
\end{itemize}