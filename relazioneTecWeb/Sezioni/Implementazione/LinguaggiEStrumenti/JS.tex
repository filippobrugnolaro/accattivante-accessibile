Il linguaggio \textit{JavaScript} è stato utilizzato principalmente per due scopi.

Il primo riguarda la validazione dei form, i controlli sono stati fatti sia nella parte pubblica che nella parte privata (utente e admin). Per ogni pagina che necessitasse di una validazione di un form è stato creato il suo corrispondente file javascript (nomeValidation.js). Lo script analizza tutti gli input previsti per il form, controllandone la validità attraverso un'espressione regolare.
    
Il secondo, invece, è molto importante e utile per aggiornare costantemente il tipo di scelte che un utente può fare in base alle disponibilità dell'impianto. Vi sono infatti casi in cui le informazioni non possono essere rappresentate staticamente. Uno di questi casi riguarda la prenotazione di un ingresso o un corso da parte dell'utente. Viene data la possibilità all'utente di noleggiare una moto (l'attrezzatura non è un problema in questi casi) ed è quindi facile intuire che la disponibilità delle moto a noleggio varia di data in data (in base alle prenotazioni degli altri utenti).

Per poter costantemente aggiornare le moto disponibili in una giornata viene fatta una chiamata AJAX ad uno script PHP che, fornita una data di apertura, restituisce le moto ancora disponibili in quella data. Questa chiamata viene eseguita ogni volta che l'utente cambia la data in cui vuole prenotare l'ingresso o il corso.

Un altro caso simile accade nel form di prenotazione dei corsi, ma per una diversa motivazione. In questo caso si utilizza una chiamata AJAX per recuperare la descrizione del corso selezionato, evitando di costringere l'utente a muoversi tra le pagine per ricordare le informazioni del corso a cui si vuole prenotare. Questo, seppur non strettamente necessario, migliora l'esperienza utente e gli consente di portare a termine le operazioni in modo più rapido.
 